\documentclass[12pt, spanish]{article}
\usepackage[spanish]{babel}
\selectlanguage{spanish}
%\usepackage{natbib}
\usepackage{url}
\usepackage[utf8x]{inputenc}
\usepackage{graphicx}
\graphicspath{{images/}}
\usepackage{parskip}
\usepackage{fancyhdr}
\usepackage{vmargin}
\usepackage{multirow}
\usepackage{float}
\usepackage{chngpage}

%Para poder hacer diagramas BNF en LaTeX
\usepackage{backnaur}
\usepackage{syntax}

\usepackage{amsfonts}

\usepackage{subcaption}

\usepackage{hyperref}
\usepackage[
    type={CC},
    modifier={by-nc-sa},
    version={4.0},
]{doclicense}

\hypersetup{
    colorlinks=true,
    linkcolor=blue,
    filecolor=magenta,
    urlcolor=cyan,
}

% para codigo
\usepackage{listings}
\usepackage{xcolor}



%% configuración de listings

\definecolor{listing-background}{HTML}{F7F7F7}
\definecolor{listing-rule}{HTML}{B3B2B3}
\definecolor{listing-numbers}{HTML}{B3B2B3}
\definecolor{listing-text-color}{HTML}{000000}
\definecolor{listing-keyword}{HTML}{435489}
\definecolor{listing-identifier}{HTML}{435489}
\definecolor{listing-string}{HTML}{00999A}
\definecolor{listing-comment}{HTML}{8E8E8E}
\definecolor{listing-javadoc-comment}{HTML}{006CA9}

\lstdefinestyle{eisvogel_listing_style}{
  language         = python,
%$if(listings-disable-line-numbers)$
%  xleftmargin      = 0.6em,
%  framexleftmargin = 0.4em,
%$else$
  numbers          = left,
  xleftmargin      = 0em,
 framexleftmargin = 0em,
%$endif$
  backgroundcolor  = \color{listing-background},
  basicstyle       = \color{listing-text-color}\small\ttfamily{}\linespread{1.15}, % print whole listing small
  breaklines       = true,
  frame            = single,
  framesep         = 0.19em,
  rulecolor        = \color{listing-rule},
  frameround       = ffff,
  tabsize          = 4,
  numberstyle      = \color{listing-numbers},
  aboveskip        = 1.0em,
  belowskip        = 0.1em,
  abovecaptionskip = 0em,
  belowcaptionskip = 1.0em,
  keywordstyle     = \color{listing-keyword}\bfseries,
  classoffset      = 0,
  sensitive        = true,
  identifierstyle  = \color{listing-identifier},
  commentstyle     = \color{listing-comment},
  morecomment      = [s][\color{listing-javadoc-comment}]{/**}{*/},
  stringstyle      = \color{listing-string},
  showstringspaces = false,
  escapeinside     = {/*@}{@*/}, % Allow LaTeX inside these special comments
  literate         =
  {á}{{\'a}}1 {é}{{\'e}}1 {í}{{\'i}}1 {ó}{{\'o}}1 {ú}{{\'u}}1
  {Á}{{\'A}}1 {É}{{\'E}}1 {Í}{{\'I}}1 {Ó}{{\'O}}1 {Ú}{{\'U}}1
  {à}{{\`a}}1 {è}{{\'e}}1 {ì}{{\`i}}1 {ò}{{\`o}}1 {ù}{{\`u}}1
  {À}{{\`A}}1 {È}{{\'E}}1 {Ì}{{\`I}}1 {Ò}{{\`O}}1 {Ù}{{\`U}}1
  {ä}{{\"a}}1 {ë}{{\"e}}1 {ï}{{\"i}}1 {ö}{{\"o}}1 {ü}{{\"u}}1
  {Ä}{{\"A}}1 {Ë}{{\"E}}1 {Ï}{{\"I}}1 {Ö}{{\"O}}1 {Ü}{{\"U}}1
  {â}{{\^a}}1 {ê}{{\^e}}1 {î}{{\^i}}1 {ô}{{\^o}}1 {û}{{\^u}}1
  {Â}{{\^A}}1 {Ê}{{\^E}}1 {Î}{{\^I}}1 {Ô}{{\^O}}1 {Û}{{\^U}}1
  {œ}{{\oe}}1 {Œ}{{\OE}}1 {æ}{{\ae}}1 {Æ}{{\AE}}1 {ß}{{\ss}}1
  {ç}{{\c c}}1 {Ç}{{\c C}}1 {ø}{{\o}}1 {å}{{\r a}}1 {Å}{{\r A}}1
  {€}{{\EUR}}1 {£}{{\pounds}}1 {«}{{\guillemotleft}}1
  {»}{{\guillemotright}}1 {ñ}{{\~n}}1 {Ñ}{{\~N}}1 {¿}{{?`}}1
  {…}{{\ldots}}1 {≥}{{>=}}1 {≤}{{<=}}1 {„}{{\glqq}}1 {“}{{\grqq}}1
  {”}{{''}}1
}
\lstset{style=eisvogel_listing_style}


\usepackage[default]{sourcesanspro}

\setmarginsrb{2 cm}{1 cm}{2 cm}{2 cm}{1 cm}{1.5 cm}{1 cm}{1.5 cm}

\title{Práctica 1. Grupo Viernes 3.\\
  \hspace{0.05cm} }
\author{Antonio David Villegas Yeguas\\
		Juan Emilio Martinez Manjon\\
		Alejandro Manzanares Lemus\\
		Najib}
\date{\today}

\renewcommand*\contentsname{hola}

\makeatletter
\let\thetitle\@title
\let\theauthor\@author
\let\thedate\@date
\makeatother

\pagestyle{fancy}
\fancyhf{}
\rhead{\theauthor}
\lhead{\thetitle}
\cfoot{\thepage}

\begin{document}

%%%%%%%%%%%%%%%%%%%%%%%%%%%%%%%%%%%%%%%%%%%%%%%%%%%%%%%%%%%%%%%%%%%%%%%%%%%%%%%%%%%%%%%%%

\begin{titlepage}
    \centering
    \vspace*{0.3 cm}
    \includegraphics[scale = 0.50]{ugr.png}\\[0.7 cm]
    %\textsc{\LARGE Universidad de Granada}\\[2.0 cm]
    \textsc{\large 4º CSI 2020/21 - Grupo 2}\\[0.5 cm]
    \textsc{\large Grado en Ingeniería Informática}\\[0.5 cm]
    \rule{\linewidth}{0.2 mm} \\[0.2 cm]
    { \huge \bfseries \thetitle}\\
    \rule{\linewidth}{0.2 mm} \\[1 cm]

    \begin{minipage}{0.4\textwidth}
        \begin{flushleft} \large
            \emph{Autores:}\\
            \theauthor\\
            \end{flushleft}
            \end{minipage}~
            \begin{minipage}{0.4\textwidth}
            \begin{flushright} \large
            \emph{Asignatura: \\
            Procesadores de Lenguajes}   \\

        \end{flushright}
    \end{minipage}\\[0.5cm]

    {\large \thedate}\\[0.5cm]
    %{\url{https://github.com/advy99/AA/}}
    {\doclicenseThis}

    \vfill

\end{titlepage}

%%%%%%%%%%%%%%%%%%%%%%%%%%%%%%%%%%%%%%%%%%%%%%%%%%%%%%%%%%%%%%%%%%%%%%%%%%%%%%%%%%%%%%%%%

\tableofcontents
\pagebreak

%%%%%%%%%%%%%%%%%%%%%%%%%%%%%%%%%%%%%%%%%%%%%%%%%%%%%%%%%%%%%%%%%%%%%%%%%%%%%%%%%%%%%%%%%



\section{Introducción: Descripción del lenguaje asignado.}

Esta práctica tratará sobre el desarrollo de un lenguaje de programación estructurado. Este lenguaje utilizará identificadores que deberán ser declarados antes de ser utilizados y tendrá como tipos de datos mínimos los tipos \texttt{entero}, \texttt{real}, \texttt{caracter} y \texttt{booleano}.

Para los tipos de datos \texttt{entero} y \texttt{real} podremos realizar las operaciones de suma, resta, producto, división y operadores de relación, mientras que para el tipo de dato \texttt{booleano} podremos realizar la operación and, or, not y xor.

El lenguaje poseerá la sentencia de asignación para todos los tipos de expresiones, además de permitir las expresiones aritméticas lógicas, además de estructuras de control básicas (\texttt{if-then-else} y \texttt{while}).

Dispondrá de una sentencia de entrada y otra de salida en la que utilizará el teclado y la salida de terminal respectivamente. La sentencia de entrada deberá permitir leer sobre una lista de identificadores y la de salida debe permitir escribir una lista de expresiones y/o constantes de tipo cadena. Estas sentencias no representarán una llamada a un subprograma.

El lenguaje asignado (BAAAC) añade también las siguientes características:

\begin{itemize}
	\item Sintaxis inspirada en C.
	\item Palabras reservadas en castellano.
	\item Listas como estructura de datos como tipo elemental. Deben considerarse las constantes de tipo lista.
	\item Funciones como subprogramas.
	\item Repeat-until como estructura de control.
\end{itemize}

\newpage

\section{Descripción de formal de la sintaxis del lenguaje usando BNF.}

\vspace{5 mm}

La sintaxis usando BNF y añadiendo las consideraciones correspondiente al lenguaje \textbf{BAAAC} sería:

\begin{bnf*}
	\bnfprod{Programa}
		{\bnfpn{Cabecera\_programa} \bnfsp \bnfpn{bloque}}\\
	\bnfprod{bloque}
		{\bnfpn{Inicio\_de\_bloque}} \\ 			
		 \bnfmore{\bnfpn{Declar\_de\_variables\_locales}} \\	
		 \bnfmore{\bnfpn{Declar\_de\_subprogs}} \\		
		 \bnfmore{\bnfpn{Sentencias}} \\	
		 \bnfmore{\bnfpn{Fin\_de\_bloque}} \\		 		 			 		
	\bnfprod{Declar\_de\_subprogs}
		{\bnfpn{Declar\_de\_subprogs} \bnfsp \bnfpn{Declar\_subprog}}\\
	\bnfprod{Declar\_subprog}
		{\bnfpn{Cabecera\_subprograma} \bnfsp \bnfpn{bloque}}\\
	\bnfprod{Declar\_de\_variables\_locales}
		{\bnfpn{Marca\_ini\_declar\_variables}}\\
		 \bnfmore{\bnfpn{Variables\_locales}} \\	
		 \bnfmore{\bnfpn{Marca\_fin\_declar\_variables}} \\		
	\bnfprod{Cabecera\_programa}
		{\bnfpn{Cabecera\_tipo\_C}}\\	
	\bnfprod{Inicio\_de\_bloque}
		{\bnfpn{Llaves\_abren}}\\	
	\bnfprod{Fin\_de\_bloque}
		{\bnfpn{Llaves\_cierran}}\\	
	\bnfprod{Variables\_locales}
		{\bnfpn{Variables\_locales} \bnfsp \bnfpn{Cuerpo\_declar\_variables}}\\	
		\bnfmore{\bnfpn{Cuerpo\_declar\_variables}} \\	
	\bnfprod{Cuerpo\_declar\_variables}
		{\bnfpn{Declar\_variables\_en\_C}}\\
	\bnfprod{Cabecera\_subprog}
		{\bnfpn{Cabecera\_funcion\_C}}\\	
	\bnfprod{Sentencias}
		{\bnfpn{Sentencias} \bnfsp \bnfpn{Sentencia}}\\	
		 \bnfmore{\bnfpn{Sentencia}} \\	
	\bnfprod{Sentencia}
		{\bnfpn{bloque}}\\	
		 \bnfmore{\bnfpn{sentencia\_asignacion}} \\	
		 \bnfmore{\bnfpn{sentencia\_if}} \\	
		 \bnfmore{\bnfpn{sentencia\_while}} \\	
		 \bnfmore{\bnfpn{sentencia\_entrada}} \\	
		 \bnfmore{\bnfpn{sentencia\_salida}} \\
		 \bnfmore{\bnfpn{sentencia\_return}} \\	
		 \bnfmore{\bnfpn{sentencia\_while}} \\	
		 \bnfmore{\bnfpn{sentencia\_repeat\_until}} \\
	\bnfprod{sentencia\_asignacion}
		{\bnfpn{operando} \bnfsp = \bnfsp \bnfpn{operando}}\\	
	\bnfprod{sentencia\_if}
		{if \bnfsp \bnfpn{condicion} \bnfsp \bnfpn{Inicio\_de\_bloque}}\\	
		 \bnfmore{\bnfpn{Cuerpo\_de\_bloque}} \\
		 \bnfmore{\bnfpn{Fin\_de\_bloque}} \\
	\bnfprod{sentencia\_while}
		{while \bnfsp \bnfpn{condicion} \bnfsp \bnfpn{Inicio\_de\_bloque}}\\	
		 \bnfmore{\bnfpn{Cuerpo\_de\_bloque}} \\
		 \bnfmore{\bnfpn{Fin\_de\_bloque}} \\
	\bnfprod{sentencia\_entrada}
		{\bnfpn{nomb\_entrada} \bnfsp \bnfpn{lista\_variables}}\\	
	\bnfprod{sentencia\_salida}
		{\bnfpn{nomb\_salida} \bnfsp \bnfpn{lista\_expresiones\_o\_cadena}}\\
	\bnfprod{expresion}
		{(\bnfpn{expresion})}\\	
		 \bnfmore{\bnfpn{op\_unario} \bnfsp \bnfpn{expresion}} \\
		 \bnfmore{\bnfpn{expresion} \bnfsp \bnfpn{op\_binario} \bnfsp \bnfpn{expresion}} \\
		 \bnfmore{\bnfpn{identificador}} \\
		 \bnfmore{\bnfpn{constante}} \\
		 \bnfmore{\bnfpn{funcion}} \\
\end{bnf*}


\section{Definición de la semántica en lenguaje natural.}

\section{Identificación de lostokenscon el máximo nivel de abstracción.}




\newpage

\section{Referencias, material y documentación usada}


\begin{thebibliography}{9}



\end{thebibliography}

\end{document}
